\documentclass[12pt]{article}
\usepackage[fleqn]{amsmath}
\usepackage{amssymb}
\usepackage{amsthm}
\usepackage{amssymb}
\usepackage{tikz}
\usepackage{tipa}
\usepackage{hyperref}
\usepackage{mathtools}
    \hypersetup{colorlinks=true,citecolor=blue,urlcolor =black,linkbordercolor={1 0 0}}
\newcommand*\circled[1]{\tikz[baseline=(char.base)]{
    \node[shape=circle,draw,inner sep=2pt] (char) {#1};}}
\newcommand{\BR}{\mathbb R}
\newcommand{\BN}{\mathbb N}
\newcommand{\prm}{^\prime}
\newcommand{\doubleprime}{^{\prime\prime}}
\newcommand{\phii}{\varphi}
\title{Lecture 13}
\begin{document}
\maketitle
\vspace*{-0.25in}
\begin{center}
	Anders Sundheim \\
	\href{mailto:asundheim@wisc.edu}{{\tt asundheim@wisc.edu}}
\end{center}
\section*{Theorem}
If $U$ is convex, and $(f_j)_{x_i}$ in $U$ for all $i,j$ \\
then $f$ is a gradient vector field
\subsection*{Proof}
\begin{proof}
  Lets just do it for $n=2$, that is, in two dimensions \\
  Fix $z\in U$. Define \\
  \[ \Psi(x)=\int_{[zx]}^rf\cdot d\alpha \]
  \[ \alpha(t)=z+t(x-z)\text{ for } 0\leq t \leq 1 \]
  \begin{align*}
    \Psi(\alpha) & =\int_0^1f(\alpha(t))\cdot \alpha\prm(t)dt \\
    & = \int_0^1f(z+t(x-z))\cdot(x-z)dt \\
  \end{align*}
  Claim: $\nabla\Psi=f$ \\
  Lets just show: $\Psi_{x_1}=f_1$ \\
  \begin{align*}
    \Psi_{x_1}(x) & =\frac{d}{dx_1}\Big(\int_0^1f(z+t(x-z))\cdot(x-z)dt\Big) \\
    & = \frac{d}{dx_1}\Big(\int_0^1f_1(z+t(x-z))\cdot(x_1-z_1)+f_2(z+t(x-z))\cdot(x_2-z_2)dt\Big) \\
    & = \int_0^1t(f_1)_{x_1}(\sim)(x_1-z_1)+f_1(\sim)+tf_2(f_2)_{x_1}(\sim)\cdot(x_2-z_2)dt \\
    & = \int_0^1t\nabla f_1(\sim)\cdot(x-z)+f_1(\sim)dt \\
    & = \int_0^1\frac{d}{dt}(tf_1(z+t(x-z)))=f_1(z+x-z)=f_1(x)
  \end{align*}
\end{proof}
\end{document}
