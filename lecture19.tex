\documentclass[12pt]{article}
\usepackage[fleqn]{amsmath}
\usepackage{amssymb}
\usepackage{amsthm}
\usepackage{amssymb}
\usepackage{tikz}
\usepackage{pgfplots}
    \pgfplotsset{width=10cm,compat=1.9}
\usepackage{tipa}
\usepackage{hyperref}
\usepackage{mathtools}
    \hypersetup{colorlinks=true,citecolor=blue,urlcolor =black,linkbordercolor={1 0 0}}
\newcommand*\circled[1]{\tikz[baseline=(char.base)]{
    \node[shape=circle,draw,inner sep=2pt] (char) {#1};}}
\newcommand{\BR}{\mathbb R}
\newcommand{\BN}{\mathbb N}
\newcommand{\prm}{^\prime}
\newcommand{\doubleprime}{^{\prime\prime}}
\newcommand{\phii}{\varphi}
\title{Lecture 19}
\begin{document}
\maketitle
\vspace*{-0.25in}
\begin{center}
	Anders Sundheim \\
	\href{mailto:asundheim@wisc.edu}{{\tt asundheim@wisc.edu}}
\end{center}
\section*{Greene's Theorem}
\subsection*{Setting}
    Let $C$ be a piecewise $C^1$ closed Jordan Curve, that is, \\
    $C$ is not self-intersecting.  $C$ encloses a region $S\subset\BR^2$ \\
\subsection*{Theorem} [Greene's Theorem] Let $P,Q:\rightarrow\BR$ be \\
    $C^1$ functions(real valued). Then, \\
    \[ \text{(*) }\oint_CP\,dx+Q\,dy=\iint_S\bigg(\frac{dQ}{dx}-\frac{dP}{dy}\bigg)\,dx\,dy \]
\subsection*{Observations}
    \circled{1} In the theorem, there are 2 functions $P$ and $Q$ that are not related \\
    $\Rightarrow$ To prove (*), it's equivalent to show 2 simpler claims. \\
    \[
        \begin{cases}
            (**)\,\oint_CP\,dx=\iint_S-\frac{dP}{dy}\,dx\,dy \\
            (***)\,\oint_CQ\,dy=\iint_S\frac{dQ}{dx}\,dx\,dy \\
        \end{cases}
    \]
    (or, you can let $P\equiv 0, Q\equiv 0$ in (*)) \\
    (**) and (***) are more or less the same \\
    \circled{2} Let's prove (**). $\oint_CP\,dx=\iint_S-\frac{dP}{dy}\,dx\,dy$ \\
    \underline{Key pt.} Divide $S$ into simple domains (in $x$), and we'll show \\
    that for each simple domain $S_k$: \\
    \[ \oint_{C_k}P\,dx=\iint_{S_k}-\frac{dP}{dy}\,dx\,dy \]
    ($C_k$ is the corresponding boundary of $S_k$). \\
    Afterwards, adding all these together in $k$ to get:
\subsection*{Proof of (**) for simple domains (in $x$)}
    \begin{proof}
        \[ S=\big\{(x,y): a\leq x\leq b, \phii(x)\leq y\leq Psi(x)\big\}\]
        We now compute easily: \\
        \[ \iint_S-\frac{dP}{dy}\,dx\,dy=\int_a^b\bigg(\int_{\phii(x)}^{\Psi(x)}-\frac{dP}{dy}\,dy\bigg)\,dx\]
        \[ = \int_a^b(P(x,\phii(x))-P(x,\Psi(x)))\,dx\]
        Let's compute $\oint_CP\,dx$: \\
        $C$ is broken into 4 pieces as following. \\
        \circled{I} $\alpha(x)=(x,\phii(x)): a\leq x\leq b, dx=dx$ \\
        \[ \int_{\alpha}=\int_a^bP(x,\phii(x))\,dx\]
        \circled{II} $\beta(s)=(b,s):\phii(x)\leq s\leq\Phi(x), dx=0$ as it's constant \\
        \[ \int_{\beta}P\,dx=0 \]
        Similarly, $\int_{\delta}P\,dx=0.$ \\
        \circled{III} $\gamma(x)=(x,\Psi(x)): a\leq x\leq b, dx=x$ \\
        \[ \int_{\gamma}P\,dx=\int_a^bP(x,\Psi(x))\,dx=-\int_a^bP(x,\Psi(x))\,dx \]
        Sum all up, \\
        \[ \oint_CP\,dx=\int_a^b\big(P(x,\phii(x))-P(x,\Psi(x))\big)\,dx \]
    \end{proof}
\subsection*{Remark}
    To prove the other identity: \\
    \[ \oint_CQ\,dy=\iint_S\frac{dQ}{dx}\,dx\,dy\text{, we simply break $S$ into simple domains in $y$} \]
\subsection*{Example}
    Compute work done by force field \\
    \[ f(x,y)=(y+3x, 2y-x) \]
    in a moving particle counter-clockwise once around the ellipse \\
    \[ 4x^2+y^2=4 \]
    \underline{Earlier}: Parameterize $C$: \\
    \[ \alpha(s)=(\cos(s),2\sin(s)), 0\leq s\leq 2\pi \]
    \[\text{Work done} = \int f\cdot d\alpha=\int_0^{2\pi}f(\alpha(s))\cdot\alpha\prm(s)\,ds \]
    \[ =\int_0^{2\pi}(2\sin(s)+3\cos(s),4\sin(s)-\cos(s))\cdot(-\sin(s),2\cos(s))\,ds \]
    \[ = \int_0^{2\pi}(-2\sin^2(s)-3\cos(s)\sin(s)+8\sin(s)\cos(s)-2\cos^2(s))\,ds \]
    \[ = \int_0^{2\pi}(-2+5\sin(s)\cos(s))\,ds =\dots \]
    \underline{The other way}: $P(x,y)=y+3x,Q(x,y)=2y-x$ \\
    Work done $=\oint_CP\,dx+Q\,dy$ \\
    \[ = \iint_S(\frac{dQ}{dx}-\frac{dP}{dy}) \]
    \[ = \iint_S(-2)\,dx\,dy \]
\end{document}
