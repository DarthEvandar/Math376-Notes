\documentclass[12pt]{article}
\usepackage[fleqn]{amsmath}
\usepackage{amssymb}
\usepackage{amsthm}
\usepackage{amssymb}
\usepackage{mathtools}
\usepackage{tikz}
\newcommand*\circled[1]{\tikz[baseline=(char.base)]{
            \node[shape=circle,draw,inner sep=2pt] (char) {#1};}}
\usepackage{hyperref}
    \hypersetup{colorlinks=true,citecolor=blue,urlcolor =black,linkbordercolor={1 0 0}}

\newcommand{\BR}{\mathbb R}
\newcommand{\doubleprime}{^{\prime\prime}}
\newcommand{\phii}{\varphi}
\newenvironment{statement}[1]{\smallskip\noindent\color[rgb]{0.50,0.00,1.00} {\bf #1.}}{}
\title{Lecture 3}
\begin{document}
\maketitle
\vspace*{-0.25in}
\begin{center}
	Anders Sundheim \\
	\href{mailto:asundheim@wisc.edu}{{\tt asundheim@wisc.edu}}
\end{center}
\section{Characterization of Critical Points}
\subsection{Critical Points in $n$ dimensions}
\[ \text{Recall} \]
\[ \phii(t) = f(y+te) \]
\[ \phii^\prime(t)=f^\prime(y+te,e)=\nabla f(y+te)e \]
\begin{align*}
  \phii\doubleprime(t) & = f\doubleprime(y+te,e,e) \\
   & =\frac{d}{dt}(\sum_{i=1}^{n}f_{x_i}(y+te)e_i) \\
   & = \sum_{i,j=1}^{n}f_{x_ix_j}(y+te)e_ie_j \\
   & = eH(y+te)e^T \\
   & =
   \begin{pmatrix}
    e_1 \\
    \vdots \\
    e_n \\
  \end{pmatrix}
  \begin{pmatrix}
   f_{x_1x_1} & \dots & f_{x_1x_n} \\
   \vdots & & \vdots \\
   f_{x_nx_1} & \dots & f_{x_nx_n} \\
  \end{pmatrix}
\end{align*}
\subsection{Hessian}
The Hessian of $f$ at $x$ is denoted by \\
\[
  H(x) = \begin{pmatrix}
  f_{x_1x_1}(x) & \dots & f_{x_1x_n}(x) \\
  \vdots & & \vdots \\
  f_{x_nx_1}(x) & \dots & f_{x_nx_n}(x)
  \end{pmatrix}
\]
$H(x)$ is a $n\times n$ matrix. Since we assume that \\
$f_{x_ix_j}(x)$ is continuous in $B(y,r)$, we have \\
$f_{x_ix_j}(x)=f_{x_jx_i}(x)\forall x,i,j$ \\
$\Rightarrow H(x)$ is a real, symmetric matrix \\
Key Point: This matrix is diagonalizable \\
\subsection{Theorem 1: Taylor expansion of $f$}
Assume f as previous \\
Then for any $h\in\BR^n$ with $|h|<r$ \\
\[ f(y+h) = f(y) + \nabla f(y)h + \frac{1}{2}h^TH(y)h+\omega(h)|h|^2 \]
\[ \text{where }\lim_{h\rightarrow0}\omega(h)=0 \]
\subsection{Characterize Critical points}
Assume $f$ has a critical point at $y$, that is $\nabla f(y) = \vec{0}$ \\
$\Rightarrow$ then for $|h|<r$ \\
\[ f(y+h) = f(y) + \frac{1}{2}h^TH(y)h+\omega(h)|h|^2 \]
\begin{align*}
  \text{Define } Q(h) & = h^TH(y)h \\
  & = \sum_{i,j=1}^{n}f_{x_ix_j}(y)h_ih_j
\end{align*}
(If $n=1$, $Q(h)=f\doubleprime(y)h^2$)
\subsection{Theorem 2: Characterization of critical points 1}
\circled{1} If $Q(h)>0$ for $h\neq 0$, then $f$ has a local min at $y$ \\
\circled{2} If $Q(h)<0$ for $h\neq 0$, then $f$ has a local max at $y$ \\
\circled{3} If we can find $h_1,h_2\neq 0$ such that $Q(h_1)>0>Q(h_2)$, then y is a saddle point
\subsection{Theorem 3: Characterization of critical points 2}
\circled{1}$\iff$ all eigenvalues of $H(y)$ are positive \\
\circled{2}$\iff$ all eigenvalues of $H(y)$ are negative \\
\circled{3}$\iff H(y)$ has at least one positive and negative eigenvalue \\
If some eigenvalues are 0, it's typically inconclusive
\subsection{Examples}
\begin{statement}{1}

\end{statement}
\[ f: \BR^2 \rightarrow\BR \]
\[ (x,y)\mapsto f(x,y)=x^2+2x+y^2-4y \]
\[ \text{Question: Characterize all critical points of }f \]
\[ \text{Solution: Find all critical points by solving} \]
\[ \nabla f(x,y) = \vec{0} = (0,0) \]
\[ (2x+2, 2y-4) = (0,0)\Rightarrow(x,y)=(-1,2) \]
\[ H(-1,2) =
\begin{pmatrix}
  f_{xx} & f_{xy} \\
  f_{yx} & f_{yy}
\end{pmatrix}
= \begin{pmatrix}
  2 & 0 \\
  0 & 2
\end{pmatrix}
\]
\[\Rightarrow\text{ all eigenvalues positive}\Rightarrow (-1,2)\text{ is a local min} \]
\\
\\
\\
\\
\\
\\
\begin{statement}{2}

\end{statement}
\[ g: \BR^2\rightarrow\BR \]
\[ (x,y)\mapsto g(x,y)=xy \]
\[ \nabla g(x,y)=(y,x)=\vec{0} \]
\[ (y,x) = (0,0)\Rightarrow\text{ one critical point at }(0,0) \]
\[
H(0,0) =
\begin{pmatrix}
  f_{xx} & f_{xy} \\
  f_{yx} & f_{yy}
\end{pmatrix}
=
\begin{pmatrix}
  0 & 1 \\
  1 & 0
\end{pmatrix}
\]
\begin{align*}
  \text{Eigenvalues} & = \text{det} (\lambda I-\begin{pmatrix} 0 & 1 \\ 1 & 0 \end{pmatrix}) \\
  & = \text{det}\begin{pmatrix} \lambda & -1 \\ -1 & \lambda \end{pmatrix} \\
  & = \lambda^2 - 1 = 0 \\
  & = \lambda^2 = 1 \\
  & = \lambda = 1,-1 \\
\end{align*}
\[ \Rightarrow \text{One positive, one negative eigenvalue}\Rightarrow(0,0)\text{ is a saddle point} \]
\end{document}
