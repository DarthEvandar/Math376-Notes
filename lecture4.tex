\documentclass[12pt]{article}
\usepackage[fleqn]{amsmath}
\usepackage{amssymb}
\usepackage{amsthm}
\usepackage{amssymb}
\usepackage{tikz}
\usepackage[utf8]{inputenc}
\usepackage[english]{babel}
\usepackage{mathtools}
\usepackage{hyperref}
    \hypersetup{colorlinks=true,citecolor=blue,urlcolor =black,linkbordercolor={1 0 0}}
\newcommand*\circled[1]{\tikz[baseline=(char.base)]{
    \node[shape=circle,draw,inner sep=2pt] (char) {#1};}}
\newcommand{\BR}{\mathbb R}
\newcommand{\doubleprime}{^{\prime\prime}}
\newcommand{\phii}{\varphi}
\title{Lecture 4}
\begin{document}
\maketitle
\vspace*{-0.25in}
\begin{center}
	Anders Sundheim \\
	\href{mailto:asundheim@wisc.edu}{{\tt asundheim@wisc.edu}}
\end{center}
\section*{Characterization of Critical Points 2}
\subsection*{Recall}
Let $A$ be a $n\times n$ real, symmetric matrix, and
$A=\begin{psmallmatrix}
a_{11} & \dots & a_{1n} \\
\vdots & & \vdots \\
a_{n1} & \dots & a_{nn}
\end{psmallmatrix}$ \\
For $h\in\BR^n$, denote by \\
\[ Q(h) = h^TAh=\sum_{i,j=1}^{n}a_{ij}h_ih_j \]
\subsection{Lemma}
\fbox{$Q(H)>0$ for all $h\neq\vec{0}\iff$ all eigenvalues of $A$ are positive}
\begin{proof}
  Since $A$ is a real, symmetric matrix, we have that $A$ is diagonalizable, \\
  and we can write \\
  \begin{equation*}
    \begin{cases}
      A = CDC^T \\
      D \text{ is a diagonalizable matrix, }D=\text{diag}(\lambda_1,\dots,\lambda_n)=\begin{pmatrix*}\lambda_1 & & 0 \\ & \ddots & \\ 0 & & \lambda_n \end{pmatrix*}\\
      CC^T = I\text{, that is, }C^T=C^{-1}
    \end{cases}
  \end{equation*}
Then, eigenvalues of A are $\lambda_1,\lambda_2,\dots,\lambda_n$ \\
we note then that:
\[ Q(h) = h^TAh = h^TCDC^Th=(h^TC)D(C^Th) \]
Set $x=C^Th\in\BR^n\Rightarrow x^T=(C^Th)^T=h^TC$. Then \\
\[ Q(h) = x^TDx=(x_1,\dots,x_n)\begin{pmatrix*}
  \lambda_1 & & 0 \\
  & \ddots & \\
  0 & & \lambda_n
\end{pmatrix*}
\begin{pmatrix*}
  x_1 \\
  \vdots \\
  x_n
\end{pmatrix*}
=\fbox{$\sum\limits_{i=1}^{n}\lambda_ix_i^2$}\]
The last identity is very important as it confirms what we need \\
\begin{itemize}
  \item Firstly, if $\lambda_1,\dots,\lambda_n>0$, then for $h\neq\vec{0},$ $x=C^Th\neq\vec{0}$, and so\\$Q(h)=\sum\limits_{i=1}^{n}\lambda_ix_i^2>0$
  \item Secondly, in order for $Q(h)>0$ for all $h\neq\vec{0}$, we need\\$\sum\limits_{i=1}^{n}\lambda_ix_i^2>0$ for all $x\neq\vec{0}$.
\end{itemize}
Clearly we yield that $\lambda_1,\dots,\lambda_n>0$
\end{proof}
\subsection{Second derivative test for critical points of functions of two variables}
\begin{equation*}
  \begin{cases}
    \text{Let }B(y,r)\subset\BR^2\text{, and }f:B(y,r)\rightarrow\BR \\
    \text{Assume that }\frac{d^2f}{d_{x_i}d_{x_j}}\text{ exists and continuous in }B(y,r)\text{ for all }1\leq i,j\leq 2 \\
    \text{Assume further that $f$ has a critical point at $y$, that is, }\fbox{$\nabla f(y)=\vec{0}$}
  \end{cases}
\end{equation*}
Our Hessian at $y$ is: \\
\fbox{
  \begin{equation*}
    H(y) =
    \begin{pmatrix*}
      f_{x_1x_1}(y) & f_{x_1x_2}(y)\\
      f_{x_2x_1}(y) & f_{x_2x_2}(y)
    \end{pmatrix*}
    =
    \begin{pmatrix*}
      A & B \\
      B & C
    \end{pmatrix*}
  \end{equation*}
}, \\
where we set $A=f_{x_1x_1}(y)$, $B=f_{x_1x_2}(y)$, $C=f_{x_2x_2}(y)$ \\
To find eigenvalues of $H(y)$, we solve \\
\[ \text{det}(\lambda I-H(y))=\text{det}
\begin{pmatrix*}
  \lambda-A & -B \\
  -B & \lambda-C
\end{pmatrix*}
=(\lambda-A)(\lambda-C)-B^2=0 \]
\[ \iff \fbox{\[ \lambda^2-(A+C)\lambda+AC-B^2=0 \]} \]
Let $\lambda_1$, $\lambda_2$ be the two roots of the above quadratic equation. \\
\begin{equation}
  \text{Then (by Vieta's theorem) }
  \begin{cases}
    \lambda_1+\lambda_2=A+C \\
    \lambda_1\cdot\lambda_2=AC-B^2=\Delta=\text{det }H(y)
  \end{cases}
\end{equation}
Then we have the following conclusions \\
\circled{1} If \fbox{$\Delta<0$}, then $\lambda_1$, $\lambda_2$ have opposite signs $\Rightarrow$ \fbox{y is a saddle point} \\
\circled{2} If \fbox{$\Delta>0$}, then $\lambda_1$, $\lambda_2$ have the same signs. \\
We now just need to determine their signs \\
Note that \\
\[ \Delta = AC-B^2>0\Rightarrow AC>B^2\geq 0\Rightarrow\fbox{$AC>0$} \]
\[ \circled{2.1}\text{ }\fbox{If $A>0$}\Rightarrow C>0\Rightarrow A+C>0\Rightarrow\lambda_1,\lambda_2>0 \]
\[ \text{which gives that }\fbox{y is a local min} \]
\[ \circled{2.2}\text{ }\fbox{If $A<0$}\text{ Same logic gives }\fbox{y is a local max} \]
\end{document}
