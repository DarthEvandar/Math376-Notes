\documentclass[12pt]{article}
\usepackage[fleqn]{amsmath}
\usepackage{amssymb}
\usepackage{amsthm}
\usepackage{amssymb}
\usepackage{tikz}
\usepackage{hyperref}
\usepackage{mathtools}
    \hypersetup{colorlinks=true,citecolor=blue,urlcolor =black,linkbordercolor={1 0 0}}
\newcommand*\circled[1]{\tikz[baseline=(char.base)]{
    \node[shape=circle,draw,inner sep=2pt] (char) {#1};}}
\newcommand{\BR}{\mathbb R}
\newcommand{\doubleprime}{^{\prime\prime}}
\newcommand{\phii}{\varphi}
\title{Lecture 5}
\begin{document}
\maketitle
\vspace*{-0.25in}
\begin{center}
	Anders Sundheim \\
	\href{mailto:asundheim@wisc.edu}{{\tt asundheim@wisc.edu}}
\end{center}
\section*{Example}
\[ f:\BR^3\rightarrow\BR \]
\[ (x,y,z)\mapsto f(x,y,z)=\vec{0} \]
\subsection*{Critical Points}
\[ \nabla f(x,y,z)=\vec{0} \]
\[ \iff (f_x,f_y,f_z)=(0,0,0) \]
\[ \iff (yz,xz,xy)=(0,0,0) \]
\[ xy = yz = zx = 0 \]
\[ (x,0,0)\forall x\in\BR \]
\[ (0,y,0)\forall y\in\BR \]
\[ (0,0,z)\forall z\in\BR \]
\[ (0,0,0) \]
\begin{align*}
  H(x,y,z) & =
  \begin{pmatrix*}
    f_{xx} & f_{xy} & f_{xz} \\
    f_{yx} & f_{yy} & f_{yz} \\
    f_{zx} & f_{zy} & f_{zz}
  \end{pmatrix*} \\
  & =
  \begin{pmatrix*}
    0 & z & y \\
    z & 0 & x \\
    y & x & 0
  \end{pmatrix*}
\end{align*}
\subsection*{At the origin}
\[ H(\vec{0}) =
\begin{pmatrix*}
  0 & 0 & 0 \\
  0 & 0 & 0 \\
  0 & 0 & 0
\end{pmatrix*}
\]
All eigenvalues are 0 $\Rightarrow$ inconclusive. \\
But we can see that $\vec{0}$ is a saddle point as for \\
any $r>0\in B(\vec{0},r)$, we take $(\frac{r}{2},\frac{r}{2},\frac{r}{2})$, $(-\frac{r}{2},-\frac{r}{2},-\frac{r}{2})$ \\
\subsection*{At the point $(0,0,1)$}
\[ H(0,0,1) =
\begin{pmatrix*}
  0 & 1 & 0 \\
  1 & 0 & 0 \\
  0 & 0 & 0
\end{pmatrix*}
\]
\end{document}
