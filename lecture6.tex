\documentclass[12pt]{article}
\usepackage[fleqn]{amsmath}
\usepackage{amssymb}
\usepackage{amsthm}
\usepackage{amssymb}
\usepackage{tikz}
\usepackage{hyperref}
\usepackage{mathtools}
    \hypersetup{colorlinks=true,citecolor=blue,urlcolor =black,linkbordercolor={1 0 0}}
\newcommand*\circled[1]{\tikz[baseline=(char.base)]{
    \node[shape=circle,draw,inner sep=2pt] (char) {#1};}}
\newcommand{\BR}{\mathbb R}
\newcommand{\doubleprime}{^{\prime\prime}}
\newcommand{\phii}{\varphi}
\title{Lecture 6}
\begin{document}
\maketitle
\vspace*{-0.25in}
\begin{center}
	Anders Sundheim \\
	\href{mailto:asundheim@wisc.edu}{{\tt asundheim@wisc.edu}}
\end{center}
\section*{Lagrange Multipliers}
\subsection*{Theorem: Lagrange Multiplier}
At a local max/min of $f$ on $S$, say $x_0\in S$, we have \\
$\nabla f(x_0)=\lambda\nabla g(x_0)$ for some $\lambda\in\BR$ \\
\begin{proof}
  Lets asume f has a local min $x_0\in S$ \\
  Lets take any nice/smooth curve $\delta:(-r,r)\rightarrow S$, with \\
  $\delta(0)=x_0$. We have 2 facts. \\
  \circled{1} $g(\delta(s)) = 0$ $\forall$ $s\in(-r,r)$ \\
  \circled{2} $s\mapsto f(\delta(s))$ has a local min at $s=0$ \\
  \\
  For \circled{1} : \\
  \[ g(\delta(s)) = 0 \]
  \[ \Rightarrow\frac{d}{ds}g(\delta(s))=\nabla g(\delta(s))\cdot\delta^\prime(s)=0 \]
  In particular, at $s=0$, \\
  \[ \nabla g(\delta(0))\cdot\delta^\prime(0) = \fbox{$\nabla g(x_0)\cdot\delta^\prime(0)=0$} \]
  $\Rightarrow\nabla g(x_0)\perp\delta^\prime(0)$ for any tangent vector $\delta^\prime(0)$ \\
  $\Rightarrow$ geometrically, \\
  \[ \fbox{$\nabla g(x_0)$ is a normal vector to surface $S$ at $x_0$} \]
  \\
  For \circled{2} : \\
  $s\mapsto\phii(s)=f(\delta(s))$ has a local min at $s=0$ \\
  that means that $\phii^\prime(0)=0$. \\
  And same as before, \\
  \[ \phii^\prime(0)=\nabla f(\delta(0))\cdot\delta^\prime(0)=\nabla f(x_0)\cdot\delta^\prime(0)=0 \]
  so $\nabla f(x_0)$ is another normal vector to $S$ at $x_0$
\end{proof}
\subsection*{Example 1}
\[ f,g:\BR^3\rightarrow\BR \]
Find the maximum of $f(x) = x_1^2$ \\
subject to $g(x)=x_1^2+x_2^2+x_3^2-4=0$ \\
\[ \nabla f(x0 = \lambda\nabla g(x)\text{ for }x\in S) \]
\[ (2x_1,0,0) = \lambda(2x_1,2x_2,2x_3) \]
There are two cases: \\
\begin{equation*}
  \begin{cases}
      \lambda = 0: x_1=0,x_2^2+x_3^2=4 \\
      \indent\Rightarrow (0,x_2,x_3) \text{ where } x_2^2+x_3^2=4 \\
      \lambda \neq 0: x_2=x_3=0 \\
      \indent\Rightarrow x_1=\pm 2 \\
  \end{cases}
\end{equation*}
\begin{align*}
  \text{We have } & f(0,x_2,x_3)=0 \\
  & f(\pm2,0,0)=4
\end{align*}
By comparing, we get the maximum of $f = 4$ \\
\subsection*{Example 2}
In $\BR^2$, find the minimum distance from the origin \\
to $\big\{(x_1,x_2)\in\BR^2:x_1,x_2=1\big\}$ \\
Find the minimum of $f(x)=x_1^2+x_2^2$ \\
subject to $g(x)=x_1x_2-1=0$ \\
\[ \nabla f(x) = \lambda\nabla g(x) \text{ for }x\in S \]
\[ \iff (2x_1,2x_2)=\lambda(x_2,x_1) \]
In this case, we have $\lambda\neq0$ \\
\begin{equation*}
  \begin{cases}
    2x_1 = \lambda x_2 \\
    2x_2 = \lambda x_1
  \end{cases}
  \Rightarrow\lambda^2=4\Rightarrow\lambda=\pm2
\end{equation*}
\fbox{If $\lambda=2,x_1=x_2$} \\
\end{document}
