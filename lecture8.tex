\documentclass[12pt]{article}
\usepackage[fleqn]{amsmath}
\usepackage{amssymb}
\usepackage{amsthm}
\usepackage{amssymb}
\usepackage{tikz}
\usepackage{hyperref}
\usepackage{mathtools}
    \hypersetup{colorlinks=true,citecolor=blue,urlcolor =black,linkbordercolor={1 0 0}}
\newcommand*\circled[1]{\tikz[baseline=(char.base)]{
    \node[shape=circle,draw,inner sep=2pt] (char) {#1};}}
\newcommand{\BR}{\mathbb R}
\newcommand{\BN}{\mathbb N}
\newcommand{\prm}{^\prime}
\newcommand{\doubleprime}{^{\prime\prime}}
\newcommand{\phii}{\varphi}
\title{Lecture 8}
\begin{document}
\maketitle
\vspace*{-0.25in}
\begin{center}
	Anders Sundheim \\
	\href{mailto:asundheim@wisc.edu}{{\tt asundheim@wisc.edu}}
\end{center}
\section*{Line Integrals}
$C^1$ and piecewise $C^1$ paths
\subsection*{Defintion}
$\alpha:[a,b]\rightarrow\BR^n$ is a path \\
And if $\alpha$ is continuous on $[a,b]$ we say it is a continuous path \\
If $\alpha$ is a continuous path and $\alpha\prm(t)$ is continuous in $(a,b)$ then we \\
say $\alpha$ is a $C^1$ path \\
If $\alpha$ is continuous and we can find $a_0=a<a_1<\dots<a_k$ and $a_k<b=a_{k+1}$ such that \\
$\alpha\prm(t)$ is continuous in $(a_i,a_{i+1})$ for all $0\leq i\leq k$, then we say that \\
$\alpha$ is a piecewise $C^1$ path \\
\subsection*{Definition of path integrals}
  \circled{1} Let $\alpha:[a,b]\rightarrow\BR^n$ be a $C^1$ path \\
  Let $f:\BR^n\rightarrow\BR^n$ be a continuous vector field \\
  \[ \text{Define }\int f\cdot dx=\int_a^bf(\alpha(t))\cdot\alpha\prm(t)dt \]
  In a more explicit way, \\
  \[ \alpha(t)=(\alpha_1(t),\alpha_2(t),\cdots,\alpha_n(t)) \]
  $f(x) = (f_1(x), f_2(x),\cdots,f_n(x))$ \\
  \[ \Rightarrow\int f\cdot dx=\int_a^b f(\alpha(t))\cdot\alpha\prm(t)dt=\int_a^b\sum_{k=1}^nf_k(\alpha(t))\alpha\prm_k(t))dt \]
  \circled{2} If $\alpha$ is a piecewise $C^1$, then we define \\
  \[ \int f\cdot dx=\int_{a_0}^{a_1}f(\alpha(t))\cdot\alpha\prm(t)dt+\cdots+\int_{a_k}^{a_{k+1}}f\cdot dt \]
  \circled{3} If $\alpha$ is a closed curve, that is $\alpha(a)=\alpha(b)$, then we also write \\
  \[ \int f\cdot d\alpha=\oint fd\alpha \]
  Note that there are many ways to parameterize a path/curve in $\BR^n$ \\
  \subsection*  {Concern}
    Do all parameterizations of the same path yield some line integral? \\
    \[ \int f\cdot d\alpha=\int f\cdot d\beta \]
  \subsection*{Proof}
    \begin{proof}
      Let $\alpha:[a,b]\rightarrow\BR^n$ is $C^1$, $t\mapsto\alpha(t)\in\BR^n$ \\
      Let $\beta:[c,d]\rightarrow\BR^n$ such that $s\mapsto\beta(s)=\alpha(u(s))$ for $c\leq s\leq d$ \\
      Where $u:[c,d]\rightarrow[a,b]$ is $C^1$, is increasing, and $u(c)=a, u(d)=b$ \\
      $\int f\cdot d\beta=\int_c^d f(\beta(s))\cdot\beta\prm(s)ds=\int_c^df(\beta(s))\cdot\alpha\prm(u(s))u\prm(s)ds$ \\
      Let $t=u(s)\Rightarrow dt=u\prm(s)ds\rightarrow=\int_a^b f(\alpha(t))\cdot\alpha\prm(t)dt$ \\
    \end{proof}
\end{document}
